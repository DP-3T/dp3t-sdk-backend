% Options for packages loaded elsewhere
\PassOptionsToPackage{unicode}{hyperref}
\PassOptionsToPackage{hyphens}{url}
%
\documentclass[a4paper
]{ubarticle}
\usepackage{graphicx}
\usepackage{amssymb,amsmath}
    \usepackage{parskip}
  \usepackage{tabu}
  \usepackage{longtable}
\makeatother

\usepackage{seqsplit}
\usepackage{hyperref}
\hypersetup{
  pdftitle={DPPPT API},
  hidelinks,
  pdfcreator={LaTeX via pandoc}}
\urlstyle{same} % disable monospaced font for URLs
\setlength{\emergencystretch}{3em} % prevent overfull lines
\providecommand{\tightlist}{%
  \setlength{\itemsep}{0pt}\setlength{\parskip}{0pt}}
\setcounter{secnumdepth}{-\maxdimen} % remove section numbering

\title{DPPPT API}
\date{\today}
\author{pepp-pt}

\begin{document}
\begin{titlepage}
	%\includegraphics[width=7cm]{ubique-logo.png}
	\hspace{4.3cm}
 {\raggedleft
	 \textbf{pepp-pt} \\
	%\hspace{11.5cm} Niederdorfstrasse 77 \\
%	8001 Zürich \\
\vspace{0.3cm}
	\url{https://www.pepp-pt.org/} 
\par}
	\vspace{3cm}
	{\Huge DP3T API \par}
	\vspace{1.5cm}
	{\huge Documentation \par}
	\vspace{3cm}
	{ \large \today }
	\end{titlepage}
\thispagestyle{empty}
\clearpage
\tableofcontents
\clearpage
\part{Technical Description}
\section{Introduction}
This document outlines the backend as it is. The models and requests are automatically generated. Hence, they should reflect the current live situation. We try to provide examples and description to clarify the use of the fields and responses returned.

\section{Verification of Data}
To handle heavy workload, requests are routed via a content-delivery-network (CDN). This means that we need to provide proof that the data was not modified by the CDN. We propose a Elliptic-Curve Digital Signature Algorithm using the P256 elliptic curve with a SHA-256 hashing algorithm. The P256 elliptic curve has good native support for the Apple and Android platforms to verify signatures. The public key should be available on the discovery platform and is as well included and distributed with the applications for iOS and Android.

The ensure the possibility of signature verification, the signed endpoints return an object with a signature and a data field, of which the data field contains a base64 representation of the list. In the current implementation the representation is a json of a list of keys. To improve performance of possible large decodings, we plan to switch to protobuf or something similar, which should speed up the parsing.

Since we only want to ensure that the data we are processing was indeed the data sent from the specified backend, it is sufficient to generate the signature of the content which will be processed. 

Too further improve operability, the algorithm used to generate the signature should as well be encoded within the json object, similiar to a JWK (Json web key).

\section{Google/Apple Privacy-Preserving Contact Tracing Similarities}

\part{Web Service}
\subsection{Introduction}
A test implementation is hosted on: https://demo.dpppt.org. 

This part of the documentation deals with the different API-Endpoints. Examples to fields are put into the models section~\ref{sec:Models} to increase readability. Every request lists all possible status codes and the reason for the status code.
\section{ /v1/ }
    \begin{verbatim}
    get /v1/
    \end{verbatim}
hello

\subsection{Responses}
\subsubsection{ 200 Success }
 

    
        \begin{ubresponses}{\textwidth}{|Y|}
        \ubheader{Type}\\
        \hline
             \hyperref[sec:string] { string } \\
 \hline

        \end{ubresponses}
    
\section{ /v1/exposed }
    \begin{verbatim}
    post /v1/exposed
    \end{verbatim}
Enpoint used to publish the SecretKey.

\subsection{ Request Headers }
\begin{ubparam}{\textwidth}{|H|c|Y|}
\ubheader{Field} & \ubheader{Type} & \ubheader{Description}\\
\hline
\ubheader{ User-Agent }   \textcolor{red}{\emph{*}}  &  string  & App Identifier (PackageName/BundleIdentifier) + App-Version + OS (Android/iOS) + OS-Version
 \\
\hline
\end{ubparam}

\subsection{ Request Body }
\begin{ubparam}{\textwidth}{|H|c|Y|}
\ubheader{Field} & \ubheader{Type} & \ubheader{Description}\\
\hline
\ubheader{  }   \textcolor{red}{\emph{*}}  &  ExposeeRequest  & The ExposeeRequest contains the SecretKey from the guessed infection date, the infection date itself, and some authentication data to verify the test result
 \\
\hline
\end{ubparam}
\subsection{Responses}
\subsubsection{ 200 Success }
Returns OK if successful
 

    
        \begin{ubresponses}{\textwidth}{|Y|}
        \ubheader{Type}\\
        \hline
             \hyperref[sec:string] { string } \\
 \hline

        \end{ubresponses}
    
\subsubsection{ 400 Bad Request }
Key is not base64 encoded
 


\section{ /v1/exposed/{dayDateStr} }
    \begin{verbatim}
    get /v1/exposed/{dayDateStr}
    \end{verbatim}


\subsection{ Path Parameters }
\begin{ubparam}{\textwidth}{|H|c|Y|}
\ubheader{Field} & \ubheader{Type} & \ubheader{Description}\\
\hline
\ubheader{ dayDateStr }   \textcolor{red}{\emph{*}}  &  string  & The date for which we want to get the SecretKey.
 \\
\hline
\end{ubparam}
\subsection{Responses}
\subsubsection{ 200 Success }
Returns ExposedOverview, which includes all secretkeys which were published on \emph{dayDateStr}.
 

    
        \begin{ubresponses}{\textwidth}{|Y|}
        \ubheader{Type}\\
        \hline
             \hyperref[sec:ExposedOverview] { ExposedOverview } \\
 \hline

        \end{ubresponses}
    
\subsubsection{ 400 Bad Request }
If dayDateStr has the wrong format
 



\part{Models}
All Models, which are used by the Enpoints are described here. For every field we give examples, to give an overview of what the backend expects.
\label{sec:Models}
\subsection{ ExposedOverview }
\label{sec:ExposedOverview}
\begin{ubresponses}{\textwidth}{|H|c|Y|p{3cm}|}
\ubheader{Field} & \ubheader{Type}  &\ubheader{Description}& \ubheader{Example}\\
\hline
 \ubheader{ exposed }  & \hyperref[sec:Exposee]{ Exposee[] }   & A list of all SecretKeys
 &  \seqsplit{c.f. Exposee model} \\
\hline

\end{ubresponses}

\subsection{ Exposee }
\label{sec:Exposee}
\begin{ubresponses}{\textwidth}{|H|c|Y|p{3cm}|}
\ubheader{Field} & \ubheader{Type}  &\ubheader{Description}& \ubheader{Example}\\
\hline
 \ubheader{ key }  \textcolor{red}{\emph{*}}  & \hyperref[sec:string]{ string }   & The SecretKey of a exposed as a base64 encoded string. The SecretKey consists of 32 bytes.
 &  \seqsplit{QUJDREVGR0hJSktMTU5PUFFSU1RVVldYWVpBQkNERUY=} \\
\hline
 \ubheader{ onset }  \textcolor{red}{\emph{*}}  & \hyperref[sec:string]{ string }   & The onset of an exposed.
 &  \seqsplit{2020-04-06} \\
\hline

\end{ubresponses}

\subsection{ ExposeeAuthData }
\label{sec:ExposeeAuthData}
\begin{ubresponses}{\textwidth}{|H|c|Y|p{3cm}|}
\ubheader{Field} & \ubheader{Type}  &\ubheader{Description}& \ubheader{Example}\\
\hline
 \ubheader{ value }  & \hyperref[sec:string]{ string }   & Authentication data used to verify the test result (base64 encoded)
 &  \seqsplit{TBD} \\
\hline

\end{ubresponses}

\subsection{ ExposeeRequest }
\label{sec:ExposeeRequest}
\begin{ubresponses}{\textwidth}{|H|c|Y|p{3cm}|}
\ubheader{Field} & \ubheader{Type}  &\ubheader{Description}& \ubheader{Example}\\
\hline
 \ubheader{ key }  \textcolor{red}{\emph{*}}  & \hyperref[sec:string]{ string }   & The SecretKey used to generate EphID base64 encoded.
 &  \seqsplit{QUJDREVGR0hJSktMTU5PUFFSU1RVVldYWVpBQkNERUY=} \\
\hline
 \ubheader{ onset }  \textcolor{red}{\emph{*}}  & \hyperref[sec:string]{ string }   & The onset date of the secret key. Format: yyyy-MM-dd
 &  \seqsplit{2019-01-31} \\
\hline
 \ubheader{ authData }  \textcolor{red}{\emph{*}}  & \hyperref[sec:ExposeeAuthData]{ ExposeeAuthData }   & AuthenticationData provided by the health institutes to verify the test results
 &  \seqsplit{TBD} \\
\hline

\end{ubresponses}



\end{document}
